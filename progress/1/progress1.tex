\documentclass{article}
\usepackage[utf8]{inputenc}
\usepackage[T1]{fontenc}
\usepackage[catalan]{babel}
\usepackage[vmargin=3cm]{geometry}
\usepackage{lastpage}
\usepackage{lipsum}
\usepackage{graphicx}
\usepackage{parskip}
\usepackage[hyphens]{url}
\usepackage{hyperref}
\usepackage{caption}
\usepackage{listings}

\usepackage{silence}
\WarningFilter{latex}{You have requested package}
\usepackage{../../common/listings-rust}

\Urlmuskip=0mu plus 2mu

\hypersetup{
    colorlinks=true,
    linkcolor=black,
    urlcolor=blue,
    citecolor=red,
}

\graphicspath{ {images/} }

\lstnewenvironment{code}[1][]{
   \noindent
   \lstset{
        language=Rust,
        style=colouredRust,
        inputencoding=latin1
    }
}{}

\begin{document}
\begin{titlepage}
	\newcommand{\HRule}{\rule{\linewidth}{0.4mm}} % Defines a new command for horizontal lines, change thickness here
	
	\center

    \vspace*{25px}
    % == Headings ==
	
	\textsc{\LARGE Universitat Autònoma de Barcelona}\\[1.5cm]

	\textsc{\Large Treball de Fi de Grau}\\[0.5cm]
	
	\textsc{\Large Informe de progrés I}\\[0.5cm]
	
	\HRule\\[0.4cm]
	
	{\LARGE\bfseries Disseny i implementació d'un llenguatge de programació amb LLVM}\\[0.4cm]
	
	\HRule\\[1.5cm]
	
	% == Author ==
	
	\begin{minipage}{0.5\textwidth}
		\begin{flushleft}
			\large
			\textit{Autor}\\
			\textsc{Josep Maria Domingo Catafal}
		\end{flushleft}
	\end{minipage}
	~
	\begin{minipage}{0.4\textwidth}
		\begin{flushright}
			\large
			\textit{Tutor}\\
			\textsc{Javier Sánchez Pujadas}
		\end{flushright}
	\end{minipage}
	
	% == Date == 

	\vfill\vfill\vfill % Position the date 3/4 down the remaining page
	
	{\large\today} % Date, change the \today to a set date if you want to be precise

	\vfill % Push the date up 1/4 Of the remaining page
\end{titlepage}

% --------------------------------------
% Table of contents
% --------------------------------------
\tableofcontents
\newpage

% --------------------------------------
% Body
% --------------------------------------
\section{Metodologia}
Per tal d'organitzar el projecte es va optar per una metodologia Agile/Scrum
però adaptada per a una sola persona. Principalment consisteix a crear esprints
d'una setmana. Un esprint és simplement un bloc de temps (una setmana en el
nostre cas) en el que s'han de completar un seguit de tasques. Les tasques han
de ser petites per tal d'oferir la màxima flexibilitat i s'agrupen en èpiques.
Les èpiques ens indiquen una funcionalitat que ha de tenir el llenguatge, i
totes les tasques que la conformen són les tasques necessàries per poder
desenvolupar aquesta funcionalitat.

\subsection{Eines}
Tot i ser un projecte d'una envergadura no molt gran, i format només per una
persona, és difícil organitzar-se sense fer ús de cap eina. És per això que
s'han utilitzat principalment dues eines: una per gestionar les tasques i els
terminis, i un altre per gestionar el control de versions del codi font.

\subsubsection{Gestió de les tasques} Per crear i gestionar les tasques s'ha
estat fent servir Jira, com vam comentar en el primer informe. Es va triar
aquesta eina, ja que l'he fet servir prèviament en projectes professionals i ja
la tenia per mà. Permet crear tasques i assignar-les a esprints, a més de crear
un roadmap amb les èpiques, mostrant de manera gràfica quan inicien i quan han
d'acabar. També té moltes funcions de mètriques, tot i que no es faran servir en
aquest projecte.

En el temps que ha transcorregut des de l'inici, està funcionant realment bé, i
ajuda bastant en l'organització.

\subsubsection{Control de versions del codi font} L'altra eina important és el
control de versions. És imprescindible, des del meu punt de vista, si has de
col·laborar amb més gent, però també, en projectes com aquest que són
individuals. Et permet estar al cas de tots els canvis que has fet en el
projecte, i en casos que trobis un bug, és molt més fàcil tirar enrere en el
temps per trobar on va aparèixer per primer cop.

De totes les eines de control de versions, s'està fent servir Git, ja que és
l'estàndard en el desenvolupament del software i és amb la que estic més 
familiaritzat. També s'està fent servir GitHub per allotjar el repositori, ja 
que ofereix moltes funcionalitats i a més és on es troben la gran majoria de
projectes de codi obert i disposa d'una gran comunitat. El repositori es pot 
trobar al següent enllaç: \url{https://github.com/josepmdc/craft}

\subsection{Control de qualitat}
Actualment el projecte disposa de tests automatitzats per al parser. Però ara
que el projecte està ja una mica més avançat la intenció és crear tests que 
compilin un programa sencer i comprovin que el resultat obtingut sigui el
correcte. Això facilitarà molt més la detecció d'errors de programació i ajudarà
a fer refactor sense causar nous errors.

\section{Seguiment de la planificació}

Si recordem, a l'inici del treball vam definir quines eren les èpiques que
conformarien el projecte i en quin moment es desenvoluparien. A hores d'ara
aquestes èpiques segueixen més o menys com s'havien plantejat al seu moment,
però amb algun canvi pel que fa al moment en el qual s'han desenvolupat (alguna
ha durat més de l'esperat). També s'ha creat una nova èpica per albergar tots
els bugs i tasques de manteniment i millora del projecte que van apareixent a
mesura que va avançant. 

A continuació mirarem aquestes èpiques, una per una, per veure quin és el seu
estat, i, per les que ja s'han completat, donar una mica més de detall de què
han aportat al projecte i com s'han desenvolupat a un nivell més tècnic.

\subsection{Implementació mínima de l'especificació}
\textit{\textbf{Completada}: 13/09/2022 - 16/10/2022} (5 esprints)

Aquesta èpica és la primera que es va completar i de la qual neixen la resta.
Consistia en la creació d'una base sòlida del llenguatge que permetés crear
programes simples amb operacions numèriques i control de flux.

Aquesta èpica és la més llarga de totes, ja que requereix molta feina inicial.
Va acabar una setmana més tard del previst, però tot i això no ha tingut gaire
afectació al conjunt del projecte.

El primer que es va fer va ser crear el projecte i un conjunt de funcions
d'utilitat. El projecte es divideix principalment en 4 mòduls:

\subsubsection{Mòdul principal}
S'encarrega de gestionar l'entrada de l'usuari (per exemple llegir l'arxiu a
compilar) i crida a la resta de mòduls, passant la  sortida d'un a l'entrada del
següent. Bàsicament és el conductor de la resta de mòduls.

\subsubsection{Lexer}
Aquest mòdul s'encarrega de llegir la seqüència de caràcters del codi a compilar
i transformar-lo en un conjunt de tokens. Principalment ens permet identificar
els símbols del llenguatge i alertar a l'usuari en cas que n'estigui fent servir
algun que no és vàlid. La manera com funciona és la següent:

Mentre que no hàgim arribat al final de la seqüència de caràcters, processem el
següent caràcter en la seqüència. 
Si és un parèntesi, per exemple, com que per si sol forma una seqüència vàlida,
i no pot generar-ne cap altre, creem un nou token de tipus parèntesis i l'afegim
al conjunt de sortida. Si per contra ens trobéssim un caràcter tipus '>', no
podem crear un token encara, ja que depenent del següent caràcter, podria ser un
'>=' o bé un '>'. Per tant, escanegem també el següent caràcter i en funció
d'això creem el token. 

Un cas particular que poder cal destacar, és quan ens trobem amb
un caràcter alfanumèric, ja que hem de tenir en compte més coses:

\begin{itemize}
\item
    Si és un dígit, vol dir que estem processant un nombre, i per tant hem
    d'anar avançant mentre ens anem trobant dígits. Mentre processem els dígits
    pot ser que ens trobem una punt, el que ens indicarà que és un nombre real i
    per tant generarem un token de float. Si, no té punt en canvi, generem un
    token d'int.
\item
    Si en canvi és un caràcter alfabètic, haurem d'anar avançant mentre que ens
    trobem caràcters alfanumèrics (no es permeten dígits a l'inici d'un
    identificador però si a la meitat) o guions baixos. Un cop troben un altre
    tipus de caràcter, vol dir que ja hem acabat i podem crear el token. Ara bé,
    aquest token pot ser un identificador o una paraula reservada del
    llenguatge, per tant, hem de comprovar si és una paraula reservada i crear
    un token o l'altre en funció d'això.
\end{itemize}

En general el lexer, un cop fet no s'haurà de modificar gaire, només caldrà
modificar-lo si hem d'afegir alguna paraula reservada nova el llenguatge o algun
símbol nou i que per tant no reconeix encara.

\subsubsection{Parser}

El lexer ens ha permès identificar els símbols del programa, però no ens permet
identificar si l'ordre és correcte o si segueix les normes del llenguatge (és a
dir si és gramaticalment correcte). Aquesta és la funció del parser, el qual a
partir de la seqüència de tokens, n'extraurà el significat i generarà un arbre
sintàctic, que ens indica com haurem d'executar les instruccions.

La funció del parser és fer complir la gramàtica del llenguatge. S'ha
implementat en la forma d'un \textit{recursive descent parser} el qual és
pràcticament una traducció literal de la gramàtica en forma de codi.

El que fa el parser és començar pel primer token de la seqüència que ens ha
generat el lexer. A partir d'allà va descendint de forma recursiva per totes les
produccions de la gramàtica. Per exemple, si el primer token és 'fn', el parser
començarà a explorar la producció que genera una funció. Mirarà que seguidament
de 'fn' hi hagi un identificar. Si no hi és, es retorna un error a l'usuari. Si
hi és, aleshores segueix descendint, mira que s'especifiquin correctament els
paràmetres, si hi són, el tipus del valor de retorn i finalment el cos de la
funció. Diem que és recursiu perquè pot ser que mentre estem generant, per
exemple una expressió, pot ser que aquesta estigui formada per altres
expressions, i per tant, tornarà a cridar de forma recursiva la producció
d'expressió.

Posem un exemple més il·lustratiu, amb una petita gramàtica, i com aquesta
s'implementaria en el parser. Aquesta seria la gramàtica per generar el
prototipus d'una funció (excloem el tipus de retorn de la funció, que aniria
després dels paràmetres, per simplificar):

\begin{verbatim}
<prototype>  ::= fn <id> "(" <params> ")"
<id>         ::= letter { letter | digit | "_" }
<params>     ::= "(" { <param> { , <param> } } ")"
<param>      ::= <id>: <type>
\end{verbatim}

La implementació d'aquesta gramàtica seria de la següent manera (en un
pseudocodi aproximat a Rust):

\begin{code}
fn parse_prototype() -> (Prototype, Err) {
    // we expect to find the fn keyword, else it's an error
    match current_token().kind {
        // The advance function moves to the next token
        TokenKind::Fn => advance(),
        _ => return Err("Expected fn keyword"),
    };

    // we expect to find the function name, else it's an error
    let name = match current_token().kind {
        TokenKind::Identifier => current_token().lexeme,
        _ => return Err("Expected an identifier"),
    };

    advance();

    // we call the params rule
    let params = parse_params();

    // we are done, we return a struct with the info of the prototype
    return Prototype { 
        name,
        params,
    };
}
\end{code}

\textit{Nota:} Els comentaris en els fragments de codi estan en anglès, ja que
\LaTeX donava problemes amb els accents.

És així de simple, iterem els tokens un per un i comprovem que el token que
estem revisant sigui el token que esperem trobar, en funció del que ens diu la
gramàtica. I quan el que esperen és una altra producció, simplement cridem a la
funció que implementa aquesta producció. En el nostre exemple la funció
\texttt{parse\_params}:

\begin{code}
    fn parse_params() -> (Vec<Variable>, Err) {
        // params start with an opening paren, else it's an error
        match current_token().kind {
            TokenKind::LeftParen => advance(),
            _ => return Err("Expected left paren"),
        };

        // if we find a closing paren, then we are done (no params)
        if current_token().kind == TokenKind::RightParen {
            advance();
            return [];
        }

        let params = [];

        loop {
            // we expect an identifier for the param name
            let name = match current_token().kind {
                TokenKind::Identifier => current_token().lexeme,
                _ => return Err("Expected an identifier"),
            };

            advance();

            // a colon separates the identifier from the type
            match current_token().kind {
                TokenKind::Colon => advance(),
                _ => return Err("Expected ':' after identifier"),
            };

            // finally the type
            let type = match current_token().kind {
                TokenKind::Identifier => current_token().lexeme,
                _ => return Err("Expected an identifier"),
            };

            // add it to the list of params
            params.push(Param { name, type });

            advance();

            match current_token().kind {
                // if it's a ')' we are done and we break from the loop
                TokenKind::RightParen => {
                    advance();
                    break;
                }
                // if it's a ',' then there's another param
                TokenKind::Comma => advance(),
                // else it's an error
                _ => return Err("Expected right paren or comma"),
            }
        }
    }
\end{code}

\subsubsection{Generació de codi}
En la definició del projecte vam comentar que fariem servir LLVM. Ara entrarem
en una mica més de detall per veure que és LLVM, com funciona i com ha estat
incorporat al projecte per tal de generar el codi.

LLVM va ser creat el 2003 per Chris Lattner (també creador del llenguatge de
programació Swift) i disposa del suport d'empreses com Apple (LLVM és una part
integral de XCode i de Swift per al desenvolupament d'aplicacions iOS), Google,
IBM o Intel. Actualment hi ha diversos dels principals llenguatges de
programació que en fan ús com ara C/C++ (a través del compilador Clang, una
alternativa a GCC), Rust o Swift.

LLVM és un toolchain per crear compiladors, és a dir un conjunt d'eines
diferents que ens ajuden en la implementació de compiladors, però principalment,
i el que ens interessa a nosaltres, és un back end (de fet n'és molts a la
vegada com ja veurem). Un back end és la part del compilador que genera
assemblador per alguna arquitectura en concret a partir d'una representació
intermèdia. Ara bé, en el cas de LLVM, no genera assemblador per una
arquitectura en concret, sinó que en pot generar per pràcticament totes les
arquitectures disponibles actualment. D'aquesta manera, tu, com a creador de
llenguatges de programació, només t'has de preocupar de generar la representació
intermèdia de LLVM. A partir d'aquí hi aplicarà optimitzacions i generarà
l'assemblador de l'arquitectura que l'hi indiquis. Fins i tot pot generar Web
Assembly, cosa que ens permet executar el llenguatge en navegadors web moderns.

Donat que LLVM és independent de l'arquitectura, quan generem el codi, no ens
hem de preocupar del nombre de registres, ja que disposem d'un nombre il·limitat
de registres virtuals, els quals LLVM mapejarà posteriorment als registres de
l'arquitectura corresponent.

Com hem comentat, LLVM també aplica optimitzacions al codi generat, com pot ser
eliminar codi que no es fa servir o avaluar expressions que es poden saber en
temps de compilació. Ara bé, perquè LLVM pugui fer aquestes optimitzacions, 
nosaltres haurem de generar el codi en SSA (Static single-assignment), el que vol
dir que només podem assignar valor a una variable una vegada. Si necessitem 
reassignar un valor, hem de crear una nova variable que la substitueixi. Això es
fa simplement perquè facilita molt a l'hora d'aplicar optimitzacions.

LLVM té el concepte de mòduls. Un mòdul conté tota la informació associada a un
arxiu de codi. Si tenim múltiples arxius, simplement hem de crear diferents 
mòduls i enllaçar-los.

Els mòduls contenen funcions i les funcions estan formades per instruccions, 
similars a les instruccions en assemblador.

Anem a veure un exemple d'un petit fragment de codi i com seria en la
representació intermèdia de LLVM. Tenim la següent funció  que rep dos enters i
en retorna el màxim:

\begin{code}
fn max(int a, int b) int {
    if a > b { a } else { b }
}
\end{code}

Si ho traduim a LLVM tenim el següent codi:

\begin{code}
define i32 @max(i32 %a, i32 %b) {
entry:
  %0 = icmp sgt i32 %a, %b
  br i1 %0, label %btrue, label %bfalse

btrue:
  br label %end

bfalse:
  br label %end

end:
  %retval = phi i32 [%a, %btrue], [%b, %bfalse]
  ret i32 %retval
}
\end{code}

La primera linia del codi anterior defineix una funció, la qual rep dos enters
de 32 bits. A continuació defineix una etiqueta entry. Aquestes etiquetes són
com les etiquetes dels assembladors, i podem anar-hi fent-hi salts.

En l'etiqueta 'entry', el primer que fa és comparar els dos enters, és a dir la
condició del 'if'. La paraula clau 'sgt' vol dir 'signed greater than', o, en
altres paraules, fa una comparació amb signe de més gran que. El resultat de la
comparació és un enter d'un bit. Si és true farà un salt a la label
\texttt{btrue}, sinó anirà a \texttt{bfalse}. 

En aquest cas les dues branques fan el mateix: un salt a l'etiqueta
\texttt{end}. Allà ens hi trobem amb un concepte que són els nodes phi. Els 
nodes phi venen a ser una espècia d'if invertit. En funció d'on hàgim fet el 
salt assignarem un valor o un altre a la variable \texttt{retval}. Si venim de
\texttt{btrue} s'assigna a, i si venim de \texttt{bfalse} s'assigna b.

Ara bé, per què les dues branques fan al salt a l'etiqueta \texttt{end} i
després es torna a fer un condicional en el bloc \texttt{end}? No podíem
assignar el valor de \texttt{retval} directament dins de la branca
\texttt{btrue} o \texttt{bfalse} i estalviar-nos aquest tercer condicional?
Doncs la resposta és que no, perquè aleshores estaríem generant codi que no està
en forma SSA. I com hem comentat anteriorment, LLVM ens obliga a generar codi en
forma SSA. I per això existeixen els nodes phi, per poder solucionar aquest
tipus de problemes.

\subsubsection{API de LLVM}
Està bé entendre el codi de LLVM, però generar tot aquest codi a mà pot ser una
mica molest i propens a errors. És per això que LLVM ens ofereix una API per
poder generar i compilar el codi LLVM. Aquesta API és per a C++, per això hem de 
fer servir una altra llibreria anomenada Inkwell, que ens defineix una API de 
LLVM per a Rust, fent crides a través de FFI a l'API de C++.

L'API disposa d'una sèrie de classes amb les quals interactuarem. La més 
fonamental és la classe \texttt{BasicBlock}, la qual representa els blocs que 
ens trobem dins d'una funció, és a dir el conjunt d'instruccions que es troben
dins d'una etiqueta. El conjunt d'aquests blocs, són les funcions, les quals
estan representades per la classe \texttt{Function}. I seguint aquesta lògica,
tenim la classe \texttt{Modue}, que representa els mòduls, que són un conjunt de
funcions.

Totes aquestes classes neixen d'una classe base que s'anomena \texttt{Value}. 
Aquesta classe representa qualsevol valor que pugui generar el programa, ja 
sigui una funció, un bloc, etc. Aquesta classe és útil de cara a implementar
les funcions de generació de codi del nostre compilador, ja que totes poden 
retornar un \texttt{Value}, i d'aquesta manera podem crear abstraccions que ens
faciliten el desenvolupament.

Una altra classe important és \texttt{Context}. Necessitarem una sola instància
d'aquesta classe, la qual conte un conjunt d'estructures de dades de LLVM, i que
contindrà l'estat de la compilació.

Finalment tenim la classe \texttt{Builder}, que és la que s'encarrega de generar
el codi com a tal, i al que, per tant, anirem cridant cada com que vulguem
generar una instrucció. Per exemle si volem construir la comparació del if de la
funció \texttt{max} anterior, ho fariem així: 

\texttt{Builder.CreateICmpSGT(a, b, "nom")}

El tercer paràmetre indica el nom de la variable generada, tot i això, només 
serveix perquè, a l'hora de debugar, ens sigui més fàcil trobar-ho.

La crida anterior ens generarà el següent codi (que és la línia 3 de l'exemple
complet que hem vist en l'apartat anterior):

\texttt{\%nom = icmp sgt i32 \%a, \%b}

\subsubsection{Anàlisi semàntica}
Un pas addicional que hem de fer és l'anàlisi semàntica, per així assegurar que
el codi que estem compilant és correcte, comprovant coses com que en una suma,
els dos valors siguin de tipus compatibles i que es puguin sumar. Aquesta
anàlisi es fa en el pas de la generació de codi, ja que estan molt lligats. Per
exemple, en el cas de la suma que hem comentat, en generar el codi, ens adonarem
que els dos tipus que estem sumant no són compatibles, ja que no podrem generar
un codi vàlid.

\subsubsection{Què ens ha aportat aquesta èpica?}
Un cop finalitzada aquesta èpica ja tenim un llenguatge mínim que pot fer coses
senzilles amb expressions numèriques. També tenim operacions de control de flux
en forma de if i while. Per exemple, el següent programa ja funcionaria:

\begin{code}
fn fib(n) {
    if n <= 1 {
        n
    } else {
        fib(n - 1) + fib(n - 2)
    }
}

fn print_i(n) {
    let i = 0;
    while i < n {
        print(i);
        i = i + 1;
    }
}
\end{code}

Podem veure que el llenguatge encara no l'hi indiquem el tipus dels paràmetres
i de retorn, ja que és algo que apareixerà més endavant. En aquesta èpica de 
moment treballem sempre en doubles.

\subsection{Declaració de tipus}
\textit{\textbf{Completada}: 17/10/2022 - 6/11/2022} (3 esprints)

L'objectiu d'aquesta èpica era permetre a l'usuari definir tipus propis a part
dels que venen per defecte en el llenguatge. Aquest objectiu s'ha traduït en la
implementació de structs dins del llenguatge, el qual permet definir conjunts de
dades de forma estructurada. Funcionen igual que els structs de C, per exemple.
La intenció es afegir mètodes als structs, per així poder dotar-los de
funcionalitat, similar a les classes d'altres llenguatges, però això es farà més
endavant si dona temps, ja que no és una funcionalitat imprescindible, donat que
es pot obtenir un funcionament similar al dels mètodes, passant els structs com
a paràmetre a les funcions.

Aquesta èpica també inclou les anotacions de tipus als paràmetres de les funcions
i el tipus de retorn de les funcions. De moment els tipus disponibles són i64,
f64 i els que l'usuari defineixi a partir dels structs.

Implementar els structs, com la gran majoria de noves funcionalitats afegides al
llenguatge implica modificar el lexer, el parser i la generació de codi. 

En quant al lexer no implica gaires complicacions. Simplement afegim una nova
paraula clau ("struct") i també perque detecti els punts com a caràcter vàlid,
ja que per accedir al camp d'un struct fem \texttt{struct.camp}.

Ampliar el parser, requereix una mica més de feina, però no difereix gaire del
que hem fet fins ara. Hem d'implementar en el parser les produccions per a la
definició de structs, per la instanciació i per l'accés als camps.

La part més complexa és la de generació de codi, ja que hem de tenir en compte
la distribució de les dades en la memòria. Donat que podem tenir tres tipus
diferents d'interaccions amb els structs els analitzarem per separat.

\begin{itemize}
\item \textbf{Definició}: Els structs en LLVM són simplement un conjunt
    de dades de diferents tipus una rere l'altre. Per exemple, el struct:

    \begin{code}
        struct SomeStruct {
            first: i64
            second: f64
        }
    \end{code}

    es representa de la següent manera:

    \begin{code}
        struct { i64, f64 }
    \end{code}

    Només indiquem els tipus dels camps i en quin ordre van, però no hi ha cap
    identificador. Per aquest motiu, es responsabilitat nostra, comprovar coses
    com ara que un camp no estigui duplicat, o saber en quina posició del struct
    es troba la dada que ens demana l'usuari a través de l'identificador. Per
    tant, durant el procés de compilació hem de guardar totes les definicions
    dels structs, els seus camps, i l'ordre en el qual es troben, per així poder
    saber on buscar la dada que ens demana l'usuari.

\item \textbf{Instanciació}: Per instanciar un struct, simplement hem de recórrer
    els camps que ens ha retornat el parser, un per un, generant les expressions
    de cada camp, i el resultat l'assignarem a la posició de memòria que li
    toqui segons l'índex del camp que se li ha assignat en la definició del
    struct. Aquesta assignació de memòria es fa igual que si assignéssim una
    variable, l'únic que canvia és com determinem la posició de memòria on
    guardar-hi el valor. LLVM disposa d'una instrucció que es diu GetElementPtr
    (GEP), que ens calcula la posició de memòria del camp a partir de
    l'apuntador a l'inici del struct i l'índex del camp.

\item \textbf{Accés}: L'accés és similar a la instanciació, però més simple.
    Simplement hem de tenir l'índex del camp al qual volem accedir dins del
    struct (per exemple si és el primer camp, l'índex 0), i l'apuntador a
    l'inici del struct. A partir d'aquesta informació amb la instrucció GEP
    obtenim la posició de memòria del camp.
\end{itemize}

\subsubsection{Què ens ha aportat aquesta èpica?}
Després d'aquesta èpica ja comencen a tenir un llenguatge una mica més complet,
ja que podem definir tipus propis a partir de structs i també especificar el tipus
dels paràmetres i del valor de retorn de les funcions.

\begin{code}
struct SomeStruct {
    first: i64
    second: i64
}

fn sum(a: SomeStruct, b: i64) i64 {
    a.first + a.second + b
}
\end{code}

\subsection{Collections}
\textit{\textbf{En progrés}: 31/10/2022 - 20/11/2022*} (3 esprints)

Aquesta èpica té l'objectiu d'afegir col·leccions d'elements al llenguatge. La idea
és implementar arrays (de mida fixa, conegut en temps de compilació), i també
slices (igual que els arrays, però la mida es pot conèixer en temps d'execució).

Si donés temps també seria interessant implementar arrays dinàmics.

Aquesta èpica està actualment en progrés. S'estan implementant els arrays, i un
cop acabats s'implementaran els slices. LLVM té suport per arrays, i per tant,
la seva implementació no és gaire complexa (similar a implementar structs). Pel
que fa als slices, és una mica més complicat. La idea és fer servir un struct,
on el primer camp sigui un punter al primer element i el segon camp, la longitud
del slice.

Donat que els strings també són arrays de caràcters, l'objectiu és poder 
implementar-los.

Pel que fa a arrays dinàmics, una opció seria fer ús de la funció malloc de C, i
enllaçar el programa amb libc.

\subsubsection{Què ens aportarà aquesta èpica?}
Un cop finalitzada aquesta èpica podrem declarar arrays i slices de la forma seguent:

\begin{code}
fn test(n: i64) {
    let array = [1, 2, 3, 4]; // array d'enters de mida fixa
    let slice = [i64; n]; // slice d'enters de mida n
    slice[0] = 1;
    print(array[0] + slice[0]);
    let str = "un string"; // array de u8;
}
\end{code}

\subsection{Mecanisme de gestió d'errors}
\textit{\textbf{Planificada}: 21/11/2022 - 11/12/2022} (3 esprints)

Actualment ell llenguatge no té manera de gestionar errors, a part de retornar
un enter que indiqui depenent del valor quin ha estat el resultat (com a C). 
Però això no és l'ideal, per això aquesta èpica serà per definir un sistema de
gestió d'errors una mica més elegant i no tan propens a errors.

\subsection{Syntactic sugar}
\textit{\textbf{Planificada}: 12/12/2022 - 02/01/2023} (3 esprints)

Implementació de "Syntactic sugar", per tal de simplificar operacions que
s'utilitzen freqüentment. Inclou bucles for in, match (similar a un switch),
list comprehension, etc.

\subsection{Suport per programació d'estil funcional}
\textit{\textbf{Planificada}: 03/01/2023 - 23/01/2023} (3 esprints)

Afegir suport per programació d'estil funcional, integrant funcions com ara map,
reduce, fold, etc.

\subsection{Deute tècnic}
\textit{\textbf{En progrés}: Termini indefinit} ($\infty$ esprints)

Aquesta és una nova èpica que s'ha afegit per albergar totes les tasques de 
manteniment i millora del projecte, com ara resolució de bugs. La duració 
d'aquesta tasca és indefinida, ja que constantment van apareixent bugs i coses
a millorar.

\newpage
\renewcommand\refname{Bibliografia}
\begin{thebibliography}{9}

\bibitem{llvmtut} \textit{LLVM Kaleidoscope Tutorial: Implementing a Language with LLVM}. Accedit el 25 d'octubre, 2022, des de \url{https://llvm.org/docs/tutorial}
\bibitem{llvm4plc} Rathi, M. \textit{A complete guide to LLVM for programming language creators}. Mukuls Blogs. Accedit el 2 de Novembre, 2022, des de \\\url{https://mukulrathi.com/create-your-own-programming-language/llvm-ir-cpp-api-tutorial}
\bibitem{ci} Nystrom, R. \textit{Crafting Interpreters}. Publicat Juliol, 2021. ISBN 0990582930.
\bibitem{mhlctllvm} \textit{Mapping High Level Constructs to LLVM IR}. Accedit el 20 d'octubre, 2022 des de \\\url{https://mapping-high-level-constructs-to-llvm-ir.readthedocs.io/en/latest/README.html}
\bibitem{inkwell} \textit{Inkwell Documentation}. Accedit el 4 d'octubre, 2022 des de \\\url{https://thedan64.github.io/inkwell/inkwell/index.html}
\bibitem{compilergo} Ball, T. \textit{Writing A Compiler In Go}. Publicat Agost, 2018. ISBN 398201610X.

\end{thebibliography}
\end{document}
