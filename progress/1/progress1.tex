\documentclass{article}
\usepackage[utf8]{inputenc}
\usepackage[T1]{fontenc}
\usepackage[catalan]{babel}
\usepackage[vmargin=3cm]{geometry}
\usepackage{lastpage}
\usepackage{lipsum}
\usepackage{graphicx}
\usepackage{parskip}
\usepackage{hyperref}
\usepackage{url}
\usepackage{caption}
\usepackage{listings}

\usepackage{silence}
\WarningFilter{latex}{You have requested package}
\usepackage{../../common/listings-rust}

\Urlmuskip=0mu plus 2mu

\hypersetup{
    colorlinks=true,
    linkcolor=black,
    urlcolor=blue,
    citecolor=red,
}

\graphicspath{ {images/} }

\lstnewenvironment{code}[1][]%
{
   \noindent
   \minipage{\linewidth} 
   \vspace{0.5\baselineskip}
   \lstset{language=Rust, style=colouredRust,#1}}
{\endminipage}

\begin{document}
\begin{titlepage}
	\newcommand{\HRule}{\rule{\linewidth}{0.4mm}} % Defines a new command for horizontal lines, change thickness here
	
	\center

    \vspace*{25px}
    % == Headings ==
	
	\textsc{\LARGE Universitat Autònoma de Barcelona}\\[1.5cm]

	\textsc{\Large Treball de Fi de Grau}\\[0.5cm]
	
	\textsc{\Large Informe de progrés I}\\[0.5cm]
	
	\HRule\\[0.4cm]
	
	{\LARGE\bfseries Disseny i implementació d'un llenguatge de programació amb LLVM}\\[0.4cm]
	
	\HRule\\[1.5cm]
	
	% == Author ==
	
	\begin{minipage}{0.5\textwidth}
		\begin{flushleft}
			\large
			\textit{Autor}\\
			\textsc{Josep Maria Domingo Catafal}
		\end{flushleft}
	\end{minipage}
	~
	\begin{minipage}{0.4\textwidth}
		\begin{flushright}
			\large
			\textit{Tutor}\\
			\textsc{Javier Sánchez Pujadas}
		\end{flushright}
	\end{minipage}
	
	% == Date == 

	\vfill\vfill\vfill % Position the date 3/4 down the remaining page
	
	{\large\today} % Date, change the \today to a set date if you want to be precise

	\vfill % Push the date up 1/4 Of the remaining page
\end{titlepage}

% --------------------------------------
% Table of contents
% --------------------------------------
\tableofcontents
\newpage

% --------------------------------------
% Body
% --------------------------------------
\section{Seguiment de la planificació}
A l'inici del treball, vam definir una sèrie de blocs de tasques que vam anomenar
èpiques, i cada una d'aquestes èpiques tenia un bloc de temps assignat (una sèrie
d'esprints d'una setmana de durada), en el qual es desenvoluparien les tasques. 
Si recordem les èpiques que es van definir eren les següents:

    \subsection{Implementació mínima de l'especificació}
        \textit{Completada: 13/09/2022 - 16/10/2022} (5 esprints)

        Aquesta èpica és la primera que es va completar i de la qual neixen la 
        resta. Consistia en la creació d'una base sòlida del llenguatge que 
        permetés crear programes simples amb operacions numèriques i control
        de flux.

        Aquesta èpica és la més llarga de totes, ja que requereix molta feina
        inicial. Va acabar una setmana més tard del previst, però tot i això no
        ha tingut gaire afectació al conjunt del projecte.

        El primer que es va fer va ser crear el projecte i un conjunt de funcions
        d'utilitat. El projecte es divideix principalment en 4 mòduls:

            \subsubsection{Mòdul principal}
            S'encarrega de gestionar l'entrada de l'usuari
                (per exemple llegir l'arxiu a compilar) i crida a la resta de mòduls, passant
                la  sortida d'un a l'entrada del següent. Bàsicament és el conductor de
                la resta de mòduls.

            \subsubsection{Lexer}
            Aquest mòdul s'encarrega de llegir la seqüència de
                caràcters del codi a compilar i transformar-lo en un conjunt de
                tokens. Principalment ens permet identificar els símbols del 
                llenguatge i alertar a l'usuari en cas que n'estigui fent servir
                algun que no és vàlid. La manera com funciona és la següent:

                Mentre que no hàgim arribat al final de la seqüència de caràcters,
                processem el següent caràcter en la seqüència. 

                Si és un parèntesi, per exemple, com que per si sol forma una 
                seqüència vàlida, i no pot generar-ne cap altre, creem un nou 
                token de tipus parèntesis i l'afegim al conjunt de sortida. Si 
                per contra ens trobéssim un caràcter tipus '>', no podem crear 
                un token encara, ja que depenent del següent caràcter, podria ser
                un '>=' o bé un '>'. Per tant, escanegem també el següent caràcter
                i en funció d'això creem el token. 

                Un cas particular que poder cal destacar, és quan ens trobem amb
                un caràcter alfanumèric, ja que hem de tenir en compte més coses:
                
                \begin{itemize}
                \item
                    Si és un dígit, vol dir que estem processant un nombre, i per tant 
                    hem d'anar avançant mentre ens anem trobant dígits. Mentre processem
                    els dígits pot ser que ens trobem una coma, el que ens indicarà
                    que és un nombre real i per tant generarem un token de float. Si,
                    no té coma en canvi, generem un token d'int.
                \item
                    Si en canvi és un caràcter alfabètic, haurem d'anar avançant
                    mentre que ens trobem caràcters alfanumèrics (no es permeten 
                    dígits a l'inici d'un identificador però si a la meitat) o 
                    guions baixos. Un cop troben un altre tipus de caràcter, vol
                    dir que ja hem acabat i podem crear el token. Ara bé, aquest
                    token pot ser un identificador o una paraula reservada del
                    llenguatge, per tant, hem de comprovar si és una paraula
                    reservada i crear un token o l'altre en funció d'això.
                \end{itemize}

                En general el lexer, un cop fet no s'haurà de modificar gaire,
                només caldrà modificar-lo si hem d'afegir alguna paraula reservada
                nova el llenguatge o algun símbol nou i que per tant no reconeix
                encara.

            \subsubsection{Parser}

                El lexer ens ha permès identificar els símbols del programa, 
                però no ens permet identificar si l'ordre és correcte o si segueix
                les normes del llenguatge (és a dir si és gramaticalment correcte).
                Aquesta és la funció del parser, el qual a partir de la seqüència 
                de tokens, n'extraurà el significat i generarà un arbre sintàctic,
                el qual ens indica com haurem d'executar les instruccions.

                La funció del parser és fer complir la gramàtica del llenguatge.
                S'ha implementat en la forma d'un \textit{recursive descent parser}
                el qual és pràcticament una traducció literal de la gramàtica en
                forma de codi.

                El que fa el parser és començar pel primer token de la seqüència
                que ens ha generat el lexer. A partir d'allà va descendint de 
                forma recursiva per totes les produccions de la gramàtica. Per
                exemple, si el primer token és 'fn', el parser començarà a 
                explorar la producció que genera una funció. Mirarà que seguidament
                de 'fn' hi hagi un identificar. Si no hi és, es retorna un error
                a l'usuari. Si hi és, aleshores segueix descendint, mira que 
                s'especifiquin correctament els paràmetres, si hi són, el valor
                de retorn i finalment el cos de la funció. Diem que és recursiu
                perquè pot ser que mentre estem generant, per exemple una expressió,
                pot ser que aquesta estigui formada per altres expressions, i per
                tant, tornarà a cridar de forma recursiva la producció d'expressió.

                Posem un exemple més il·lustratiu, amb una petita gramàtica, i com
                aquesta s'implementaria en el parser. Aquesta seria la gramàtica
                per generar el prototipus d'una funció (excloem el tipus de retorn
                de la funció, que aniria després dels paràmetres, per simplificar):

                \begin{verbatim}

             <prototype>  ::= fn <id> "(" <params> ")"
             <id>         ::= letter { letter | digit | "_" }
             <params>     ::= "(" { <param> { , <param> } } ")"
             <param>      ::= <id>: <type>

                \end{verbatim}

                La implementació d'aquesta gramàtica seria de la següent manera
                (en un pseudocodi aproximat a Rust):

                \begin{code}
                fn parse_prototype() -> (Prototype, Err) {
                    // we expect to find the fn keyword, else it's an error
                    match current_token().kind {
                        TokenKind::Fn => advance(),
                        _ => return Err("Expected fn keyword"),
                    };

                    // we expect to find the function name, else it's an error
                    let name = match current_token().kind {
                        TokenKind::Identifier => current_token().lexeme,
                        _ => return Err("Expected an identifier"),
                    };

                    advance();

                    // we call the params rule
                    let params = parse_params();

                    // we are done, we return a struct with the info of the prototype
                    return Prototype { 
                        name,
                        params,
                    };
                }
                \end{code}

                It's that simple, we go through all the tokens checking if the 
                token we are looking at is the one we expected acording to the
                grammar. And when we expect another rule, we just call another
                function that implements that rule. In out exemple the 
                \texttt{parse\_params} function:

                \begin{code}
                    fn parse_params() -> (Vec<Variable>, Err) {
                        // params start with an opening paren
                        match current_token().kind {
                            TokenKind::LeftParen => advance(),
                            _ => return Err("Expected left paren"),
                        };

                        // if we find a closing paren, then we are done (no params)
                        if current_token().kind == TokenKind::RightParen {
                            advance();
                            return [];
                        }

                        let params = [];

                        loop {
                            // we expect an identifier for the param name
                            let name = match current_token().kind {
                                TokenKind::Identifier => current_token().lexeme,
                                _ => return Err("Expected an identifier"),
                            };

                            advance();

                            // a colon separates the identifier from the type
                            match current_token().kind {
                                TokenKind::Colon => advance(),
                                _ => return Err("Expected ':' after identifier"),
                            };

                            // finally the type
                            let type = match current_token().kind {
                                TokenKind::Identifier => current_token().lexeme,
                                _ => return Err("Expected an identifier"),
                            };

                            advance();

                            // add it to the list of params
                            params.push(Param { name, type });

                            // if ')' we are done, if comma then there's another param
                            match current_token().kind {
                                TokenKind::RightParen => {
                                    advance();
                                    break;
                                }
                                TokenKind::Comma => advance(),
                                _ => return Err("Expected right paren or comma"),
                            }
                        }
                    }
                \end{code}

            \subsubsection{Anàlisi semàntica i generació de codi}

        Un cop construïda la base, afegir noves funcionalitats, és molt més fàcil.


    \subsection{Declaració de tipus}
        \textit{Completada: 17/10/2022 - 6/11/2022} (3 esprints)
        
        L'objectiu d'aquesta èpica era permetre a l'usuari definir tipus propis
        a part dels que venen per defecte en el llenguatge. Aquest objectiu s'ha
        traduït en la implementació de structs dins del llenguatge, el qual permet
        definir conjunts de dades de forma estructurada. Funcionen igual que els
        structs de C, per exemple. La intenció es afegir mètodes als structs, per
        així poder dotar-los de funcionalitat, similar a les classes d'altres
        llenguatges, però això es farà més endavant si dona temps, ja que no és
        una funcionalitat imprescindible, donat que es pot obtenir un funcionament
        similar al dels mètodes, passant els structs com a paràmetre a les funcions.

        Implementar els structs, com la gran majoria de noves funcionalitats
        afegides al llenguatge implica modificar el lexer, el parser i la 
        generació de codi. 

        En quant al lexer no implica gaires complicacions. 
        Simplement afegim una nova paraula clau ("struct") i també perque detecti
        els punts com a caràcter vàlid, ja que per accedir al camp d'un struct
        fem \texttt{struct.camp}.

        Ampliar el parser, requereix una mica més de feina, però no difereix 
        gaire del que hem fet fins ara. Hem d'implementar en el parser les 
        produccions per a la definició de structs, per la instanciació i 
        per l'accés als camps.

       La part més complexa és la de generació de codi, ja que hem de tenir en
       compte la distribució de les dades en la memòria. Donat que podem tenir
       tres tipus diferents d'interaccions amb els structs els analitzarem per 
       separat.

       \begin{itemize}
        \item \textbf{Definició}: Els structs en LLVM són simplement un conjunt
            de dades de diferents tipus una rere l'altre. Per exemple, el struct:

            \begin{code}
                struct {
                    first: i64
                    second: f64
                }
            \end{code}

            es representa de la següent manera:

            \begin{code}
                struct { i64, f64 }
            \end{code}

            Només indiquem els tipus dels camps i en quin ordre van, però no hi
            ha cap identificador. Per aquest motiu, es responsabilitat nostra,
            comprovar coses com ara que un camp no estigui duplicat, o saber en
            quina posició del struct es troba la dada que ens demana l'usuari a
            través de l'identificador. Per tant, durant el procés de compilació
            hem de guardar totes les definicions dels structs, els seus camps, i
            l'ordre en el qual es troben.
       \end{itemize}

    \subsection{Collections}
        \textit{En progrés: 31/10/2022 - 20/11/2022*} (3 esprints)

        Implementació d'arrays estàtics, arrays dinàmics i tuples.

    \subsection{Mecanisme de gestió d'errors}
        Implementar un mecanisme per tal de gestionar errors en temps d'execució.

    \subsection{Syntactic sugar}
        Implementació de "Syntactic sugar", per tal de simplificar operacions
        que s'utilitzen freqüentment. Inclou bucles for in, match (similar a un
        switch), list comprehension, etc.

\subsection{Suport per programació d'estil funcional}
Afegir suport per programació d'estil funcional, integrant funcions com ara map,
reduce, fold, etc.

\section{Passos a seguir}

\begin{thebibliography}{9}

\bibitem{repo} Codi del projecte, \url{https://github.com/josepmdc/craft}

\end{thebibliography}
\end{document}
