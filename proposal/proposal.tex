\documentclass{article}
\usepackage[utf8]{inputenc}
\usepackage[T1]{fontenc}
\usepackage[catalan]{babel}
\usepackage{fancyhdr}
\usepackage[head=50pt, vmargin=3cm]{geometry}
\usepackage{lastpage}
\usepackage{lipsum}
\usepackage{graphicx}
\usepackage{parskip}

\graphicspath{ {images/} }

\begin{document}
\begin{titlepage}
	\newcommand{\HRule}{\rule{\linewidth}{0.4mm}} % Defines a new command for horizontal lines, change thickness here
	
	\center

    \vspace*{25px}
    % == Headings ==
	
	\textsc{\LARGE Universitat Autònoma de Barcelona}\\[1.5cm]
	
	\textsc{\Large Treball de Fi de Grau}\\[0.5cm]
	
	\textsc{\Large Informe inicial}\\[0.5cm]
	
	\HRule\\[0.4cm]
	
	{\LARGE\bfseries Disseny i implementació d'un llenguatge de programació amb LLVM}\\[0.4cm]
	
	\HRule\\[1.5cm]
	
	% == Author ==
	
	\begin{minipage}{0.5\textwidth}
		\begin{flushleft}
			\large
			\textit{Autor}\\
			\textsc{Josep Maria Domingo Catafal}
		\end{flushleft}
	\end{minipage}
	~
	\begin{minipage}{0.4\textwidth}
		\begin{flushright}
			\large
			\textit{Tutor}\\
			\textsc{Javier Sánchez Pujadas}
		\end{flushright}
	\end{minipage}
	
	% == Date == 

	\vfill\vfill\vfill % Position the date 3/4 down the remaining page
	
	{\large\today} % Date, change the \today to a set date if you want to be precise

	\vfill % Push the date up 1/4 of the remaining page
\end{titlepage}

% --------------------------------------
% Body
% --------------------------------------
\section{Introducció i el problema a resoldre}
Historicament, generalitzant molt, podem trobar dos tipus de llenguatges: els que 
són ràpids de programar però lents d'executar (ex. Python), i els que són lents 
de programar i ràpids d'executar. En l'última decada han aparegut nous
llenguatges, com ara Go, que intenten situar-se al mig d'aquests dos paradigmes.

\section{Objectiu}
L'objectiu d'aquest treball, és dissenyar i implementar un llenguatge de 
programació que pugui ser executat de forma relativament ràpida, però al mateix 
temps que permeti al programador tenir una expèriencia el més plàcida possible.

Un altre objectiu del llenguatge és intentar minimitzar al màxim els errors que
pot cometre el programador, i per tant reduir el nombre de bugs que es poden
generar en el programa. Per això, serà un llenguatge fortament tipat i inclourà
algunes propietats de llenguatges funcionals com ara la immutabilitat per 
defecte (es pot mutar si el programador vol, però ho ha de fer de forma 
explicita). Si una funció muta l'estat de l'objecte, aquesta ho indicara de 
forma explicita cada cop que es invocada.

Per tal de que el llenguatge permeti un desenvolupament àgil, la gestió de 
memòria serà automàtica.



\section{Metodologia}
Per tal d'organitzar el desenvolupament del projecte, s'han dividit les 
diversers funcionalitats del llenguatge en grans blocs: Un bloc per tenir una 
funcionalitat bàsica i una sèrie de blocs que afegeixen funcionalitat extra.

Cada bloc després, conté un seguit de tasques a dins, que definiexen de forma
més granular, que és el que s'ha de desenvolupar.

Per tal de gestionar aquests blocs, és farà servir GitHub, que disposa d'una 
funcionalitat anomenada "Milestones". Aquesta funcionalitat permet crear blocs
de tasques ("milestones")  i posar una data límit a cada bloc. Donat que el 
codi del projecte està allotjat a GitHub, això ens permet enlleçar Pull Requests
amb les tasques, i quan es completa una PR, la tasca es completa automaticament.

Les "milestones" que s'han definit són les següents:

\begin{enumerate}
    \item \textbf{Implementació mínima de l'especificació \null\hfill \textsc{Data límit:} 30 de setembre}\\
        Implementar tota la funcionalitat basica per poder fer programes 
        numerics senzills. Inclou operacions aritmetiques, funcions, control de 
        flux bàsic (if statements i bucles while) i declaració de variables.

    \item \textbf{Type declarations \null\hfill \textsc{Data límit:} 10 d'octubre}\\
        Implementar la funcionalitat necesaria per definir tipus de dades 
        propis (structs, interficies i type aliases).

    \item \textbf{Collections \null\hfill \textsc{Data límit:} 31 d'octubre}\\
        TODO
    
    \item \textbf{Add an error handling mechanism \null\hfill \textsc{Data límit:} 27 de novembre}\\
        TODO
    
    \item \textbf{Syntactic sugar \null\hfill \textsc{Data límit:} 11 de desembre}\\
        Implementació de "Syntactic sugar", per tal de simplificar operacions 
        que s'utilitzen frequentment. Inclou bucles for in, match (similar a un
        switch), list comprehension, etc.
    
    \item \textbf{Support for functional style programming \null\hfill \textsc{Data límit:} 25 de desembre}\\
        TODO
\end{enumerate}

Cada una d'aquestes funcionalitats, generalment, es divideixen en tres 
subtasques principals: implementar el lexer, implementar el parser i 
implementar la generació de codi.

\section{Passos a seguir (planificació)}

\section{Definició del llenguatge}
\begin{itemize}
    \item Gramàtica
    \item Exemples de codi
\end{itemize}

\end{document}
